\documentclass{article}

\usepackage{math_packages}
\usepackage{user_commands}



\begin{document}
\begin{center}
    \textbf{Measure theory}
\end{center}

All the definitions that follow are directly taken from the book \cite{salamon2020measure}. We specify on which page the definition is given with brackets, for example we write $\{12\}$ for page 12.


\section{Borel measures}

\pp{81} Let $(X, \UU)$ be a topological space. It is called locally compact if each point has a compact neighborhood. It is called $\sigma$-compact if there exists an increasing sequence of compact sets $(K_i)$ such that
\begin{equation}
    \bigcup_{i \in \NN} K_i = X.
\end{equation}
We assume in this section that $(X, \UU)$ is a locally compact Hausdorff space and denote by $\BB$ the Borel $\sigma$-algebra generated by $\UU$.

\subsection{Regular Borel Measures}

\begin{flexidefinition}{D3.1 - \pp{82}}[Borel Measure, regular measure]
A measure $\mu : \BB \to [0, \infty]$ is called
\begin{itemize}
    \item a Borel measure if $\mu(K) < \infty$ for every compact set $K \subset X$
    \item outer regular if for every Borel set,
    \begin{equation}\label{def:outer regular}
        \mu(B) = \inf \{\mu(U) : B \subseteq U, U \text{ is open} \}
    \end{equation}
    \item inner regular if for every Borel set,
    \begin{equation}\label{def:inner regular}
        \mu(B) = \sup \{\mu(K) : B \subseteq K, K \text{ is compact} \}.
    \end{equation}
\end{itemize}
A Radon measure is an inner regular Borel measure.
\end{flexidefinition}
\begin{flexidefinition}{\pp{86}}[Local regularity]
    Let $\AM$ be a subclass of the Borel sets $\BB$. We say that a measure $\mu$ is outer (inner) regular for the class $\AM$ if (\ref{def:outer regular}) (or (\ref{def:inner regular})) is satisfied, with the inf (sup) taken on $\AM$.
\end{flexidefinition}
\begin{flexilemma}{L3.7 - \pp{86}}[]
    Let $\mu$ be an outer regular Borel measure that is inner regular for $\UU$. Then,
    \begin{itemize}
        \item $\mu$ is inner regular for $\{ B \in \BB : \mu(B) < \infty \}$
        \item If $X$ is $\sigma$-compact then $\mu$ is regular.
    \end{itemize}
\end{flexilemma}

\begin{flexitheorem}{T3.8 - \pp{87}}[Riez]
    Let $\mu_1 : \BB \to [0, \infty]$ be an outer regular Borel measure that is inner regular on open sets. Let 
    \begin{equation}
        \mu_0(B) := \sup \{\mu_1(K) : K \subset B \ and \ K \ is \ compact \}.
    \end{equation}
    The following holds:
    \begin{enumerate}[label = (\roman*)]
        \item $\mu_0$ is a Radon measure, it agrees with $\mu_1$ on all compact sets and all open sets, and $\mu_0(B) \leq \mu_1(B)$ for all $B \in \BB$
        \item If $X$ is $\sigma$-compact then $\mu_0 = \mu_1$,
        \item If $f: X \to \R$ is a compactly supported continuous function then
        \begin{equation}
            \int_X f d \mu_0 = \int_X f d \mu_1.
        \end{equation}
        \item Let $\mu : \BB \to [0,\infty]$ be a Borel measure that is inner regular on open sets. Then, $\int_X f d\mu = \int_X f d\mu_1$ for every compactly supported continuous function iff $\mu_0 \leq \mu \leq \mu_1$.
    \end{enumerate}
\end{flexitheorem}


\subsection{Outer Borel Measures}

\begin{flexidefinition}{D1.54 - \pp{39}}[Complete measure space]
    A measure space $(X, \AM, \mu)$ is complete if 
    \begin{equation}
        N \in \AM, \mu(N) = 0, E \subset N \Rightarrow E \in \AM.
    \end{equation}
\end{flexidefinition}

\begin{flexidefinition}{D2.3 - \pp{50}}[Outer measure]
    Let $X$ be a set. A function $\nu : 2^X \to [0, \infty]$ is called an outer measure if it satisfies the following axioms:
    \begin{enumerate}[label = (\alph*)]
        \item $\nu(\varnothing) = 0$,
        \item If $A \subset B \subset X$ then $\nu(A) \leq \nu(B)$,
        \item If $A_i \subset X$ for $i \in \N$, then $\nu\left( \cup_{i=1}^\infty A_i \right) \leq \sum_{i=1}^\infty \nu(A_i)$.
    \end{enumerate}
    A set $A$ is called $\nu$-measurable if is satisfies
    \begin{equation}
        \nu(D) = \nu(D \cap A) + \nu(D \backslash A).
    \end{equation}
\end{flexidefinition}

\begin{flexitheorem}{T2.4 - \pp{50}}[Caratheodory]
    Let $\AM$ be the set of $\nu$-measurable sets. Then, $\AA$ is a $\sigma$-algebra, $\mu := \nu_{|\AM}$ is a measure and the measure space $(X, \AA, \mu)$ is complete.
\end{flexitheorem}

\begin{flexidefinition}{D3.11 - \pp{92}}[Borel outer measure]
    A Borel outer measure on $X$ is an outer measure $\nu : 2^X \to [0, \infty]$ that satisfies the following axioms:
    \begin{enumerate}[label = (\alph*)]
        \item If $K \subset X$ is compact then $\nu(K) < \infty$.
        \item If $K_0,K_1 \subset X$ are disjoint compact sets then $\nu(K_0 \cup K_1) = \nu(K_0) + \nu(K_1)$.
        \item $\nu$ is outer regular on every set,
        \item $\nu$ is inner regular on every open set.
    \end{enumerate}
\end{flexidefinition}

\begin{flexitheorem}{T3.12 - \pp{92}}[Caratheodory for Borel outer measures]
    Let $\nu$ be a Borel outer measure. Then $\nu_\BB$ is an outer regular Borel measure and is inner regular on open sets.
\end{flexitheorem}

\subsection{Riez representation theorem}

\begin{flexidefinition}{\pp{97}}[Compactly supported function]
    A function $f : X \to \R$ is said to be compactly supported if its support 
    \begin{equation}
        \supp(f) = \overline{\left\{x \in X : f(x) \neq 0 \right\}}
    \end{equation}
    is compact. We define:
    \begin{equation}
        C_c(X) = \{f : X \to \R | f \text{ is continuous and compactly supported}\}.
    \end{equation}
\end{flexidefinition}

\begin{flexidefinition}{D3.13 - \pp{97}}[Positive linear functional]
    A linear functional $\Lambda : C_c(X) \to \R$ is said to be positive if 
    \begin{equation}
        f \geq 0 \Rightarrow \Lambda(f) \geq 0.
    \end{equation}
\end{flexidefinition}

\begin{flexilemma}{\pp{98}}[Continuous functions with bounded support are integrable]
    Let $\mu: \BB \to [0,\infty]$ be a Borel measure. For all $f \in C_c(X)$, $f$ is $\mu$-integrable.
\end{flexilemma}

We define:
\begin{equation}
    \Lambda_\mu(f) := \int_X f d\mu.
\end{equation}

\begin{flexitheorem}{T3.15 - \pp{98}}[Riez representation theorem]
    Let $\Lambda : C_c(X) \to \R$ be a positive linear functional. Then,
    \begin{enumerate}[label = (\roman*)]
        \item There exists a unique Radon measure $\mu_0$ such that $\Lambda_{\mu_0} = \Lambda$
        \item There exists a unique outer regular Borel measure $\mu_1$ that is inner regular on open sets and $\Lambda_{\mu_1} = \Lambda$
        \item The Boral measures $\mu_0$ and $\mu_1$ agree on compact and open sets, and $\mu_0 \leq \mu_1$
        \item Let $\mu$ be a Boral measure that is inner regular on open sets. Then, $\Lambda_{\mu} = \Lambda$ implies that $\mu_0 \leq \mu \leq \mu_1$.
    \end{enumerate}
\end{flexitheorem}

\begin{flexilemma}{C3.17 - \pp{105}}[Outer Radon]
    Radon measures are outer regular on compact sets.
\end{flexilemma}

\begin{flexitheorem}{T3.18 - \pp{105}}[Borel measures are regular]
    Let $X$ be a locally compact Hausdorff space. Then,
    \begin{enumerate}[label = (\alph*)]
        \item Assume $X$ is $\sigma$-compact. Then, every Borel measure that is inner regular on open sets is regular.
        \item Assume that every open set of $X$ is $\sigma$-compact. Then, every Borel measure is regular.
    \end{enumerate}
\end{flexitheorem}

\begin{flexidefinition}{\pp{106}}[Bases, countability]
    A basis of a topological space $(X, \UU)$ is a collection $\VV$ such that for every $U \in \UU$, there is a subset $\SM$ of $\VV$ such that $U = \bigcup_{V \in \SM} V$. A second countable topological space has a countable basis. For a first countable topological space, every point has a local countable basis, i.e. there is a countable sequence of sets $W_i$, that all contain $x$, such that every open set that contains $x$ contains one of the $W_i$.
\end{flexidefinition}

\begin{flexilemma}{L3.21 - \pp{106}}[$\sigma$-compactness and countability]
    Let $X$ be a locally compact Hausdorff space.
    \begin{enumerate}[label=(\roman*)]
        \item If $X$ is a second countable space, then every open subset is $\sigma$-compact.
        \item If every open set of $X$ is $\sigma$-compact, then $X$ is first countable.
    \end{enumerate}
\end{flexilemma}

The Alexandrov double arrow space is an example of a space that is $\sigma$-compact but not second countable.





\bibliography{biblio}{}
\bibliographystyle{alpha}

\end{document}